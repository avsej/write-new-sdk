\documentclass{report}
\usepackage[a4paper]{geometry}
\usepackage{fontspec}
\setmainfont{CMU Serif}
\setsansfont{CMU Sans Serif}
\setmonofont{CMU Typewriter Text}
\defaultfontfeatures{Ligatures=TeX}
\usepackage{polyglossia}
\setmainlanguage{english}
\usepackage{hyperref}

\begin{document}
\chapter{Developing a Client Library}

This chapter is relevant for developers who are creating their own client library that communicates with the Couchbase
Server. For instance, developers creating a library for language or framework that is not yet supported by Couchbase
would be interested in this content.

Couchbase SDKs provide an abstraction layer your web application will use to communicate with a cluster. Your
application logic does not need to contain logic about navigating information in a cluster, nor does it need much
additional code for handling data requests.

Once your client makes calls into your client library the following should be handled automatically at the SDK level:

\begin{itemize}
    \item maintain direct communications with the cluster,

    \item determining cluster topology,

    \item distribute requests to the cluster; automatically make reads and writes to the correct node in a cluster,

    \item direct and redirect requests based on topology changes,

    \item handle and direct requests appropriately during a failover.
\end{itemize}

The cluster exposes its services using several protocols. For management and query APIs the client must be able to
communicate HTTP protocol with JSON encoding for payloads (in most of the cases, but in several places more simple
URL-form encoding might be required. For data operations, like Upsert or Get, more efficient binary protocol must be
implemened. In this document we will call this protocol MCBP (stands for Memcache Binary Protocol). The protocol derives
from memcache binary protocol, although it does not aim to be compatible with existing memcache client libraries.

This document uses OCaml code snippets, because at the time of writing there is no supported SDK for this language,
therefore the implementation will be free from legacy implementation decisions. For the same reason we assume that the
cluster exposes APIs that became available in Couchbase Server 7.0 and don't cover any deprecated features.

\section{Bootstrap process}

This section describes a number of different behaviours within a client.  We first start by discussing how individual
MCBP connections should be established and authenticated.  We then further describe the overall connection behaviour for
an entire cluster, the process for connecting to a particular bucket, along with a number of required optimizations to
these processes to enhance the applications performance in various scenarios.

\subsection{MCBP Connection Sequence}

Individual MCBP connections are described as being "connected" only once the connection has been completely initialized
at a protocol level (\texttt{HELLO}, \texttt{ERRMAP}, \texttt{AUTH}). A connection can be either a cluster connection or
a bucket connection, the differentiation solely being whether a \texttt{SELECT\_BUCKET} has yet been performed on the
connection. Note that fetching a configuration from a connection is NOT considered part of the initialization of a
memcached connection and is explicitly discussed outside the scope of the memcached initialization, though in the case
of the pipelining that is performed, a valid operation may be sent as part of the initial connection pipeline sequence
in order to reduce new-connection operational round trip time (RTT) to 1.

All MCBP connections should first establish an appropriate network connection to the target node via TCP. Once the
connection is established, the client must perform a sequence of commands to initialize the connection. This sequence
must be performed in the order given in order to allow correct behaviour. This sequence MUST be dispatched in a single
batch (except in the case of \texttt{SASL\_CONTINUE}) without waiting for responses. The entire batch of initialization
commands will be handled in sequence upon receiving responses, as described further below in this section.

\begin{description}
    \item[\texttt{HELLO}]
        A HELLO operation must be dispatched to the node with a list of requested features along with a client
        and connection identifiers as defined in "Client Connection ID".
\end{description}

\section{Unified Client Agent}

The SDKs send an actual HTTP \texttt{User-Agent} header to services like query or analytics, but they also need to send a
trimmed-down version via a MCBP \texttt{HELLO} negotiation.

\begin{verbatim}
cb-<SDK-Identifier>/<Version> (<System-Information>)
\end{verbatim}

\subsection{SDK-Identifier}

Mandatory field. The identifier allows to uniquely distinguish between the different SDKs used. The following identifiers must be
followed by the implementations:

\begin{itemize}
    \item .NET: \texttt{dotnet}
    \item Go: \texttt{golang}
    \item NodeJS: \texttt{nodejs}
    \item C: \texttt{lcb}
    \item PHP: \texttt{php}
    \item Python: \texttt{python}
    \item Java: \texttt{java}
    \item Scala: \texttt{scala}
    \item Kotlin: \texttt{kotlin}
    \item JVM Core-IO: \texttt{jvm-core}
    \item Ruby \texttt{ruby}
    \item Rust \texttt{rust}
\end{itemize}

We are going to use \texttt{ocaml}, because the document uses OCaml as example implementation.

\subsection{Version}

Mandatory field. The version must include the significant digits of the current followed SemVer standard and can
optionally include a platform-dependent suffix.

\begin{verbatim}
major.minor.patch[suffix]
\end{verbatim}

If for whatever reason a version cannot be determined, \texttt{0.0.0} must be used.

\subsection{System-Information}

Optional field. The system information is the most flexible piece since it will inevitably be different for each
platform. Nonetheless a structure like the following must be followed to help with automated parsing:

\begin{verbatim}
(<os>; <platform>)
\end{verbatim}

More entries might be added with \texttt{;} (semicolon) as separator, but we recommend including OS and Platform
descriptions as first two entries. Note that if system infromation is not  available, the surrounding braces must be
omitted completely. If only one entry is available, the \texttt{;} separator must be omitted.

\begin{description}
    \item[os]
        The OS identifier, maybe including the architecture.  For example: \texttt{Unix/64}
    \item[platform]
        If the SDK is running in a platform/runtime, include information. This could also include application server
        information. For example: \texttt{OCaml/4.11.1}
\end{description}

\section{Client Connection ID}

Each client instance must have unique identifier. In addition, it must generate unique identifiers for every network
connection.

Each MCBP connection MUST serialize both client and connection identifiers using the following JSON format, and send it
as a key with \texttt{HELLO} command during connection sequence. The server will associate the given uuid to the socket
and use it when logging slow operations and tracing data.

\begin{description}
    \item [\texttt{a} (string)]
        The unified client agent string --- this can not be longer than 200 characters and must be trimmed if exceeds
        200.
    \item [\texttt{i} (string)]
        The connection ID is be made up of two components separated by a forward slash. The first value will be
        consistent for all connections in a given client instance, and the second value is per connection. Each part is
        a randomly generated uint64 number that is formatted as a hex string and padded to 16 characters.
\end{description}

The example key is below

\begin{verbatim}
{
  "a":"cb-ocaml/1.0.0 (Unix; OCaml/4.11.1)",
  "i":"000ae6bd34d5ee12/48e7bce204c4e9ad"
}
\end{verbatim}

\end{document}
